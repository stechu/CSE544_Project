\section{Methodology}
    
\subsection{Systems that we compare}
We choose 3 different big data systems in different categories. 

\textbf{Myria.} Myria 
\cite{HalperinACCKMORWWXBHS14SIGMOD, myriaurl} is a 
new shared nothing parallel relational database management system (RDBMS) 
with focus on complex workflows from science. It uses relational data model
and embraces many modern technologies and architectural decisions which are 
widely used in big data systems. We choose Myria as the 
representative of state of art distributed relational big data systems.

\textbf{Spark.} Spark \cite{ZahariaCDDMMFSS12NSDI} is a distributed 
computing engine with significant
performance improve over Hadoop MapReduce. It has an DAG execution
engine that support cyclic data flow and in memory computing. We choose Spark 
as the representative of state of art MapReduce like system.


\textbf{GraphLab.} GraphLab \cite{GonzalezLGBG12OSDI} is a distributed graph 
computation engine for real world large scale graphs. We choose GraphLab
as the representative of state of art graph processing system.

\subsection{Metrics}

The advantages brought by big data system are the ability of process large 
amount of data using high level abstraction. So we measure two metrics:

\emph{Lines of code (LOC).} One advantage of modern big data systems 
is to allow users to write high level code without worrying details of 
data communication. Thus, we use lines of code as a measure of degree
of abstraction of the DSL that these system uses. For fairness, we 
define the following rules:
\begin{enumerate}
    \item Calling application specific libraries is not allowed. 
    \item Each line of the program should not exceed 80 characters.
    \item Comment does not count into LOC. 
\end{enumerate}

\emph{Runtime.} We deployed three big data system in Amazon EC2. 
For each system,
we use the same number (1 masters, 16 workers) and the same type (m3.large)
of EC2 instances. The runtime does not include data loading time. 
