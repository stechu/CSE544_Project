
% ----------------------------------------------------------------------------------------------------------------
% This .tex file (and associated .cls V3.2SP) *DOES NOT* produce:
%       1) The Permission Statement
%       2) The Conference (location) Info information
%       3) The Copyright Line with ACM data
%       4) Page numbering
% ---------------------------------------------------------------------------------------------------------------

%% A bit of setup for convenient switching between draft and production modes
\def\draft{1}
%% And the commands themselves
\newcommand{\ifdraft}[0]{\ifnum\draft=1}
\RequirePackage[final]{graphicx}

\ifdraft
\documentclass[draft,pdftex,letterpaper]{acm_proc_article-sp}
\newcommand{\PAGENUMBERS}{yes}       % "yes" or "no"
\else
\documentclass[pdftex,letterpaper]{acm_proc_article-sp}
\newcommand{\PAGENUMBERS}{no}       % "yes" or "no"
\fi


\usepackage{balance}  % to better equalize the last page
\usepackage{times}    % comment if you want LaTeX's default font
\usepackage{url}      % llt: nicely formatted URLs
\usepackage{tabularx}
\usepackage{float}
%\usepackage{color}
\usepackage{url}
\usepackage{algpseudocode}
\usepackage{algorithm}
\usepackage{verbatim}
\usepackage{mathtools}
\usepackage{caption}
\usepackage{subcaption}
%\usepackage{amsmath}
\let\proof\relax
\let\endproof\relax
\usepackage{amsthm}
\usepackage{thmtools}
\usepackage{xspace}
\usepackage{multirow}


%%%
%%%  Comments
%%%

\usepackage[usenames,dvipsnames]{color}

\ifdraft
\newcommand{\note}[2]{
    \textbf{\textcolor{#1}{#2}}\xspace
}
\else\newcommand{\note}[2]{\unskip}
\fi


%\newtheorem{lemma}{Lemma}[section]
%\theoremstyle{definition}
%\newtheorem{definition}{Definition}[section]
%\newtheorem{example}{Example}[section]

\declaretheorem[numberwithin=section]{definition}

\newcommand{\ie}{{\em i.e.}\xspace}
\newcommand{\eg}{{\em e.g.}\xspace}
\newcommand{\cf}{{\em c.f.}\xspace}
\newcommand{\ea}{{\em et al.}\xspace}
\newcommand{\aka}{{\em a.k.a.}\xspace}

\providecommand{\e}[1]{\ensuremath{\times 10^{#1}}}
\newenvironment{packed_item}{
\begin{itemize}
   \setlength{\itemsep}{1pt}
   \setlength{\parskip}{0pt}
   \setlength{\parsep}{0pt}
}
{\end{itemize}}

\newcommand\xqed[1]{%
  \leavevmode\unskip\penalty9999 \hbox{}\nobreak\hfill
  \quad\hbox{#1}}
\newcommand\marker{\xqed{$\square$}}

\newenvironment{packed_enum}{
\begin{enumerate}
   \setlength{\itemsep}{1pt}
   \setlength{\parskip}{0pt}
   \setlength{\parsep}{0pt}
}
{\end{enumerate}}

\renewcommand{\algorithmicrequire}{\textbf{Input:}}
\renewcommand{\algorithmicensure}{\textbf{Output:}}

\newdef{example}{\textit{Example}}

\begin{document}

\title{Benchmark Big Data Systems on Complex Analytic Queries}

% \numberofauthors{1}
% \author{
% \alignauthor
%  author1$^\dag$, author2$^{\dag,\S}$, author3$^\dag$\\
%        \affaddr{$^\dag$Computer Science and Engineering Department, University of Washington}\\
%        \affaddr{Seattle, Washington, USA}\\
%        \affaddr{$^\S$PETROBRAS S.A., Rio de Janeiro, RJ, Brazil}\\
%        \email{\{email1, email2, email3 \}@cs.washington.edu}
% }

\thispagestyle{empty}
\setcounter{page}{0}
\newpage


\maketitle

\begin{abstract}

In recent years, considerable number of big data systems emerged for
 large-scale data analyis following MapReduce. 
Most of the systems can handle the big volume of data if the analytic query 
itself is not complex.
In this paper, we would like to study how these systems perform against complex
queries. We consider the query is complex if the query has the following 
properties: 1) the query is iterative and cannot be done in a single round
of communication. 2) the query contains aggregates or UDFs which are widely
used for data analytics. 3) the query processes significant amount of data.
We proposed a benchmark consisting of a collection of such queries and
evaluated these queries on state-of-art representive big data systems in
its catagory. We think this emprical evaluation and analysis could be 
beneficial to the development of next generation big data systems.

\end{abstract}

%\category{H.2.4}{Database Management}{Systems}

%\terms{Management, Performance}

\begin{sloppypar}

\section{Introduction}

Following MapReduce, many big data analytics systems emerges in recent years, 
including Spark-SparkSQL \cite{XinRZFSS13SIGMOD, ZahariaCDDMMFSS12NSDI}, 
GraphLab \cite{GonzalezLGBG12OSDI}, Myria~\cite{HalperinACCKMORWWXBHS14SIGMOD} 
and others \cite{AbouzeidBARS09PVLDB,ThusooSJSCZALM10ICDE}. One similarity 
among these systems is that they all deploy in shared-nothing cluster and can 
scale well for relatively simple queries over large data due to massive 
parallelism. 

In this paper, we ask a vital question, how do these system perform if the
query is ``complex''. We characterize the query as complex if the query has 
the following properties.

\begin{enumerate}

\item \textbf{Iterative}. This means that the query requires some
iterative computation. Example can be like PageRank and graph reachability. 

\item \textbf{Aggregation and Filtering}. Aggregation means the final
result of the query is aggregated and contains possibly much less data
than input. This can be think of aggregation in SQL or \texttt{reduce} on data
by applying combining function. Data filtering means the input data needs to be
filtered by certain predicate of conditions. Both aggregation and filtering is 
not abnormal in data curating or ETL process.

\item \textbf{Multiple data sources}. This requires the input data of the 
the query is from more than one data source or tables. This requirement is not
rare in practice and will require the system to handle data communication
properly since in many cases the system need to send data between servers.

\end{enumerate}

We choose this criteria from two different aspects of considerations. On one
 hand, these properties are not abnormal in analytic processes.
  For example, to 
get value from data, many algorithms like PageRank, K-Means require iterative
query. Also data analysts usually will spend a large portion of time to do ETL
or data curating, which requires the ability of aggregation and filtering. And 
in real world applications, data can usually comes from different data 
sources or tables and need to be combined together. On the other hand, these
properties requires clever system design and implementation to be efficiently
computed. And simple parallelization may not need to satisfactory 
performance. For example, pipelining hadoop jobs to execute iterative 
queries will suffer from Hadoop's huge cost of serialization and 
deserialization and the cost of synchronization between jobs. From these two 
perspective, we pick these properties to make this benchmark practical yet 
could be helpful to design future big data systems. 

XXXXXXXXXXXXXXXXXXXXXXXXXXXXXXXXX
Discussion based on experiments
XXXXXXXXXXXXXXXXXXXXXXXXXXXXXXXXX



\section{Methodology}
    
\subsection{Systems that we compare}

\subsection{Metrics}

\section{Benchmark Queries}
\label{sec:benchmark}

\subsection{Q1: LCA}
This query is to compute the least common ancestor (LCA) of two 
academic papers. An ancestor of paper $p$ is transitively defined as:

\begin{enumerate}
    \item Any paper that $p$ cites is an ancestor of $p$
    \item Any paper that $p$'s ancestor cites is an ancestor of $p$
\end{enumerate} 

Given two paper $p_1$ and $p_2$, if their sets of ancestors are 
$a(p_1)$ and $a(p_2)$, their common ancestors are 
$a(p_1) \cap a(p_2)$. We define the distance from a common ancestor
as $d(a_1, p_1, p_2) = max(dist(a_1, p_1), dist(a_1, p_2))$, where 
$dist(a_1, p_1)$ is the minimum number of citation hops from $a_1$ to 
$p_1$. Then we can define a total ordering among the common ancestors
of $p_1$ and $p_2$ as, if $a_i, a_j \in a(p_1) \cap a(p_2)$, 
$a_i \prec a_j$ if and only if.

\begin{enumerate}
    \item $d(a_i, p_1, p_2) < d(a_j, p_1, p_2)$.
    \item $d(a_i, p_1, p_2) = d(a_j, p_1, p_2)$ and the year when $a_i$
    published $year(a_i)$ is earlier than $a_j$'s $year(a_j)$.
    \item if $d(a_i, p_1, p_2) = d(a_j, p_1, p_2)$ and $year(a_i) = year(a_j)$, $a_i$'s paper id is a smaller number than $a_j$'s.
\end{enumerate}

\subsection{Q2: K-Core}

The query is to compute the $k$-core (or $k$-degenerate graph) 
of an undirected graph. It is firstly defined by Paul Erd\H{o}s 
and Hajnal as color number \cite{ErdosH66}. 
$k$-core is an important structural property
and has been studied extensively in network analytics 
\cite{Alvarez-HamelinDBV05NIPS, ChengKCO11ICDE}.

A $k$-core of a graph $G$ is a maximal induced subgraph of $G$ in which all
vertices have degree at least $k$ in the subgraph. 
Figure~\ref{fig:k-core_example} shows $1 \ldots 3$-cores of a graph. 

\begin{figure}[t]
    \centering
    \includegraphics[width=0.9\linewidth]{images/kcore.pdf}
    \caption{Example of k-cores of a graph $G$}
    \label{fig:k-core_example}
\end{figure}

$k$-core has two interesting properties. First, as showed in 
Figure~\ref{fig:k-core_example}, $n$-core contains all the vertices 
in $n+1$-core. Second, there is a polynomial time ($O(m)$, $m$ is the 
number of edges of the graph) algorithm for 
core decomposition (compute all cores) \cite{BatageljM03CORR}.
Algorithm~\ref{alg:k-core} shows the algorithm of computing $k$-core.

\begin{algorithm}
\caption{Core Decomposition Algorithm}
\label{alg:k-core}
\begin{algorithmic}[1]
\Require k \Comment{k}
\Require G \Comment{Input Graph}
\While{true}
    \State $count \leftarrow 0$
    \ForAll{every vertex $v \in V(G)$}
        \If{$deg(v) < k$}
            \State remove $v$ from $V(G)$
            \State remove edges adjacent to $v$ from $E(G)$
            \State $count \leftarrow count+1$
        \EndIf
    \EndFor
    \If{$count = 0$}
        \State break.
    \EndIf
\EndWhile
\end{algorithmic}
\end{algorithm}

\subsection{Q3: Merger-Tree}

The merger tree query is from large-scale cosmological simulation in astronomy.
The astronomers want to track the evolution of galaxies from the Big Bang
to the present day, which spans over 14 billion years. The merger tree query 
compute the hierarchical assembly of galaxies by tracking the merging of 
small galaxies. In our evaluation, we assume that all the preprocessing has
been properly done and evaluate the 3rd computation step
 in \cite{LoebmanOCOAHBQG14SIGMOD}. The query will compute weighted edges
 (number of particles shared) between two galactic groups in two adjacent time
 stamp ($t$ and $t+1$). Also, this query only compute the edges related to a 
 set of particles that specified by the astronomers (Particles of interest).

 \begin{figure}[t]
 \begin{verbatim}
 particle(pid, grp_id, time) :- 
    poi(pid, grp_id, time), 
    halo(grp_id, time, totalParticles>256).
 
 edges(time, gid1, gid2, $count(*)) :-
    particle(pid, gid1, time), 
    particle(pid, gid2, time+1).
 
 treeEdges(1, gid1, gid2, count) :- 
    edges(time=1, gid1, gid2, count).

 treeEdges(time+1, gid2, gid3, count) :-
    treeEdges(time, gid1, gid2, count), 
    edges(time+1, gid2, gid3, count). 
 \end{verbatim}
 \caption{Merger Tree Query}
 \label{fig:merger-tree}
 \end{figure}

We show the datalog version of this query in Figure~\ref{fig:merger-tree}. 
The query firstly uses halo table to filter out all the particles that are 
in the halo with total number of particles less or equal than $256$ 
(those are considered as insignificant halo by the astronomers). 
Then we computed all the edges by joining the particle particle table.
At last we compute all the tree edges, which only consider the edges that can
be traced ``back'' to current time. 

\section{Experimental Evaluation}

We report the experimental result of running three benchmark queries from 
Section~\ref{sec:benchmark}. We deployed the three systems in Amazon EC2 using 
the identical configuration, which uses 1 master node and 16 slave nodes.
Each node is a m3.large instance which has 2 vCPU, 7.5 GB RAM and 32GB SSD 
storage. We used the most up to date version of the 3 systems that we can get:
Myria (Daily built in Mar 10, 2015, commit: 90d85cd), Spark (1.2.1 release) and
GraphLab (Open sourced version in GitHub, commit: 18c2103). All our source code
is open sourced in \url{https://github.com/stechu/CSE544_Project}.

\subsection{Objective Evaluation}
\label{subsec:obje}

\subsubsection{Q1}

We evaluate the performance of three systems on Q1 using jstor scientific digital
library data. We use two tables: 
\begin{enumerate}
    \item Paper(pid:int, year:int) is a table with two columns. The first column is the unique paper id of each paper. The second column is the year
    when the paper was published. This table contains 1.8 million papers.
    \item Citation(p1:int, p2:int) is a table with two columns. Each row of 
    this paper represents a citation. The first column of this table is the
    citing paper, and the second is the cited paper. This table contains 8.2
    million citations.
\end{enumerate}

Figure~\ref{fig:q1} shows the runtime and lines of code (LOC) using three 
systems. We can observe the GraphLab has the fastest runtime among three 
systems. Myria is slightly slower (1.38x runtime) than GraphLab. Spark is 
slowest among the three systems. Its runtime is 43x GraphLab's runtime.  
In term of LOC (Figure~\ref{fig:q1_loc}), both Myria and Spark use less than 
70 lines to express this query while GraphLab need nearly 400 lines. This is 
due to the programming model and the language support. GraphLab uses C++ and 
force the programmer to ``think like a vertex''.

\begin{figure}[t]
    \centering
    \begin{subfigure}{0.7\linewidth}
        \includegraphics[width=\textwidth]{images/q1_runtime.png}
        \caption{Runtime}
        \label{fig:q1_runtime}
    \end{subfigure}
    \begin{subfigure}{0.7\linewidth}
        \includegraphics[width=\textwidth]{images/q1_loc.png}
        \caption{Lines of Code}
        \label{fig:q1_loc}
    \end{subfigure}
\caption{Q1: LCA}
\label{fig:q1}
\end{figure}

\subsubsection{Q2}

We evaluate the performance of the three system on Q2 using sampled twitter 
social network data. The table contains two columns, the first column is the 
id of follower and the second is the id of the followee. Since $k$-core
requires a undirected graph, we add an reversed edge between two nodes 
if there is a single directed edge between them. The tables contains about 
2 million rows.

Figure~\ref{fig:q2} shows the runtime and lines of code using three systems. 
We can observe that Myria has smallest runtime among all the three systems. 
GraphLab has about $1.8x$ Myria's runtime and Spark has about $5x$ Myria's 
runtime. In terms of lines of code, Myria needs only $13$ lines of code, while
Spark needs $23$ lines of code and GraphLab used $146$ lines of code. 

\begin{figure}[t]
    \centering
    \begin{subfigure}{0.7\linewidth}
        \includegraphics[width=\textwidth]{images/q2_runtime.png}
        \caption{Runtime}
        \label{fig:q2_runtime}
    \end{subfigure}
    \begin{subfigure}{0.7\linewidth}
        \includegraphics[width=\textwidth]{images/q2_loc.png}
        \caption{Lines of Code}
        \label{fig:q2_loc}
    \end{subfigure}
\caption{Q2: k-core}
\label{fig:q2}
\end{figure}

\subsubsection{Q3}

We evaluate the performance of the three systems on Q3 using $13$ snapshots of 
galaxy data from UW astronomy department. We use two tables:

\begin{enumerate}
    \item Halo(grpId:long, timeStep:long, mass:double, numParticles: long) is 
    a table with 4 columns. The first two columns form a unique identification 
    (primary key) for a halo. The third column is the mass of a halo. The
    forth column contains the number of particles within a halo. This table 
    contains 68K halos.
    \item ParticleOfInterest(pid:long, mass:double, type:string, grpId:long, 
     timeStep: long) is a table with 5 columns. The first column is the unique
     identifier (primary key) of a particle. The second column is the mass of a
     particle. The third column contains the type of a particle. The last two 
     columns decides which group (halo) a particle belongs to. This table
     contains 42 million particles.
\end{enumerate}


\begin{figure}[t]
    \centering
    \begin{subfigure}{0.7\linewidth}
      \includegraphics[width=\textwidth]{images/q3_runtime.png}
        \caption{Runtime}
        \label{fig:q3_runtime}
    \end{subfigure}
    \vspace{10pt}
    \begin{subfigure}{0.7\linewidth}

      \includegraphics[width=\textwidth]{images/q3_loc.png}
      
        \caption{Lines of Code}
        \label{fig:q3_loc}
    \end{subfigure}
    \vspace{10pt}
        \begin{subfigure}{0.7\linewidth}

      \includegraphics[width=\textwidth]{images/q3_runtime2.png}
      
        \caption{Runtime without self-join}
        \label{fig:q3_runtime2}
        \end{subfigure}
\caption{Q3: merger tree}
\label{fig:q3}
\end{figure}

Figure~\ref{fig:q3} shows the runtime and lines of code using three systems. Compared to the previous 2 queries, query 3 contains a few join operations that Myria and Spark supports really well, but Graphlab doesn't have native support of. In terms of runtime performance, similar to the previous 2 queries, Myria completed first, and Spark was orders of magnitude slower. Graphlab was not evaluated in the runtime performance due to relative difficulty of implementing inner join within its GAS programming abstraction. We later performed a separate evaluation where Myria in pitched with Graphlab for the groupby and count operations. Graphlab was 2x slower probably due to its run including some input data ingestion time, while Myria's run has already preloaded all the data into the cluster. For lines of code, continuing the trend, Myria needed only $25$ lines of code, while
Spark needs $39$ lines of code and GraphLab used $187$ lines of code.




\subsection{Comparing the three systems}
\label{sec:comp}
We try to draw some subjective conclusions for 
the three system from the objective evaluation 
in Section~\ref{subsec:obje} and from our experience of using these three systems in this section.

\subsubsection{Myria}
Myria topped two queries in runtime and topped all three query in LOC, which
clearly shows that on the scenario that we defined. Myria is fast and at a 
very high level of abstraction to make the users who writes analytic queries
easier. We think this good performance comes from two reasons:

\begin{enumerate}
    \item The relational abstraction fits well with ``Big Data''. The 
    relational data model provide user a declarative language that easy 
    to express analytic workload. On the other hand, extensive research
    on query optimization will enable efficient parallel evaluation of 
    the relational query.
    \item As a state of art distributed shared nothing database system, 
    Myria adopted many new technology such as light weight iterative 
    processing and parallel data flow based back-end engine. The good 
    system implementation contributes to the performance as well. 
\end{enumerate} 

The inconvenience of using Myria is its data ingestion process. Although the 
system support ingesting data from S3, the user have to use post a json 
formatted query to do the data ingestion while cannot direct write S3 bucket 
as the data source in the query.

\subsubsection{Spark}

Spark is a new MapReduce implementation with significant improvement on 
iterative and in-memory computing ability. We find that the most significant  
strengths of spark is its
Well matured and system and eco-system. Spark is very easy to
deploy in EC2. The programming guide and documentation is very user friendly. 

In terms of abstraction level, Spark is below Myria. In Spark python, user 
still need to use MapReduce like transform function to operate data. It is 
slightly less convenient compared with Myria. For example, to do a join, the 
the user need firstly apply a map with a function which transform the joined 
columns to key. But overall, using Spark python to express all the three query
is not hard and uses relatively small LOCs.

The biggest problem of Spark is performance. Spark is the slowest system in 2 
of 3 queries. We do not have a concrete idea on the reason for that. But our 
guess could be the synchronization barrier over iteration in spark is still 
larger than other modern systems (Myria and GraphLab), although improved 
largely from Hadoop MapReduce.

Another problem of Spark is lacking automatic control of parallelism of RDD. 
The parallelism of RDD will be the sum of the two source RDDs if user joins
them. This causes the parallelism of RDD grows exponentially if there are
joins inside an iteration. This will quickly lead to run out of system 
sources since the overhead of each RDD container is not negligible. This 
problem can be solved by forcing keeping parallelism after join. But it is
not very easy for a user to identify this problem and find the solution.   
    
\subsubsection{GraphLab}

Graphlab is a high performance, distributed graph based computing framework. Compared to the other big data frameworks, graphlab differs by exploiting the locality and parallelism of most graph analytics algorithms, such as PageRank and k-core. Graphlab allows users to write functional programs that will asynchronously execute on all the vertices. They term their programming paradigm GAS (gather, apply and scatter). Gather functions are executed on all incoming edges, which serves to gather information from neighboring vertices. Apply functions merge the information collected by the gather functions and update the data in the vertex. And finally the scatter functions will scatter the new data in the vertex to its neighbors.

Graphlab has good performance due to being implemented in C++ and leverages all the locality and parallelism that exist in most graph analytics algorithms. In some way, they have offer the similar benefits of a traditional data flow machine, where vertex program is asynchronously executed only when a neighboring vertex has gotten some new information.

However, there are 2 big problems with Graphlab. The first being the restrictiveness of its programming interface. While it is very intuitive to express graph analytic algorithms in Graphlab's GAS model, other types of complex data operations such as joining, filtering are not very straight forward to implement. Secondly, Graphlab is by far the more complex tool to use due to its complicated C++ interface, and the need to write a bunch of housekeeping code. As shown in previous sections, GraphLab requires an order of magnitude more lines of code to implement the same algorithm compared to Spark and Myria. More line of code also means the programmers are going to have a steeper learning curve and more places to make mistakes. Finally, Graphlab is also quite limited in its data ingestion modules, as it only supports loading data from local disk and HDFS. And the user has to manually partition in the input files in order for the file to be loaded in parallel.



\section{Conclusion}

We defined a benchmark which considering three important properties of modern
complex analytic queries. We evaluated three state of art big data systems 
which have different data model and programming abstraction over our benchmark
using the same instance setting in EC2. We discuss the pros and cons of these
systems and reported our experience on deploying and using these three systems.

\section{Acknowledgments}

We would like to thank Daniel Halperin for his MyriaL code on LCA and thank 
Maaz Ahmad and Antoine Kaufmann for their GraphLab code on LCA. We would like 
to thank Jennifer Ortiz and Laurel Orr for their help on myMergerTree query. 
We would like to thank Magdalena Balazinska for the helpful discussion on the
project.





\end{sloppypar}

\newpage

% \scriptsize
\bibliographystyle{abbrv}
\bibliography{report}

\end{document}
