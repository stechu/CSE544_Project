\section{Benchmark Queries}
\label{sec:benchmark}

\subsection{Q1: LCA}
This query is to compute the least common ancestor (LCA) of two 
academic papers. An ancestor of paper $p$ is transitively defined as:

\begin{enumerate}
    \item Any paper that $p$ cites is an ancestor of $p$
    \item Any paper that $p$'s ancestor cites is an ancestor of $p$
\end{enumerate} 

Given two paper $p_1$ and $p_2$, if their sets of ancestors are 
$a(p_1)$ and $a(p_2)$, their common ancestors are 
$a(p_1) \cap a(p_2)$. We define the distance from a common ancestor
as $d(a_1, p_1, p_2) = max(dist(a_1, p_1), dist(a_1, p_2))$, where 
$dist(a_1, p_1)$ is the minimum number of citation hops from $a_1$ to 
$p_1$. Then we can define a total ordering among the common ancestors
of $p_1$ and $p_2$ as, if $a_i, a_j \in a(p_1) \cap a(p_2)$, 
$a_i \prec a_j$ if and only if.

\begin{enumerate}
    \item $d(a_i, p_1, p_2) < d(a_j, p_1, p_2)$.
    \item $d(a_i, p_1, p_2) = d(a_j, p_1, p_2)$ and the year when $a_i$
    published $year(a_i)$ is earlier than $a_j$'s $year(a_j)$.
    \item if $d(a_i, p_1, p_2) = d(a_j, p_1, p_2)$ and $year(a_i) = year(a_j)$, $a_i$'s paper id is a smaller number than $a_j$'s.
\end{enumerate}

\subsection{Q2: K-Core}

The query is to compute the $k$-core (or $k$-degenerate graph) 
of an undirected graph. It is firstly defined by Paul Erd\H{o}s 
and Hajnal as color number \cite{ErdosH66}. 
$k$-core is an important structural property
and has been studied extensively in network analytics 
\cite{Alvarez-HamelinDBV05NIPS, ChengKCO11ICDE}.

A $k$-core of a graph $G$ is a maximal induced subgraph of $G$ in which all
vertices have degree at least $k$ in the subgraph. 
Figure~\ref{fig:k-core_example} shows $1 \ldots 3$-cores of a graph. 

\begin{figure}[t]
    \centering
    \includegraphics[width=0.9\linewidth]{images/kcore.pdf}
    \caption{Example of k-cores of a graph $G$}
    \label{fig:k-core_example}
\end{figure}

$k$-core has two interesting properties. First, as showed in 
Figure~\ref{fig:k-core_example}, $n$-core contains all the vertices 
in $n+1$-core. Second, there is a polynomial time ($O(m)$, $m$ is the 
number of edges of the graph) algorithm for 
core decomposition (compute all cores) \cite{BatageljM03CORR}.
Algorithm~\ref{alg:k-core} shows the algorithm of computing $k$-core.

\begin{algorithm}
\caption{Core Decomposition Algorithm}
\label{alg:k-core}
\begin{algorithmic}[1]
\Require k \Comment{k}
\Require G \Comment{Input Graph}
\While{true}
    \State $count \leftarrow 0$
    \ForAll{every vertex $v \in V(G)$}
        \If{$deg(v) < k$}
            \State remove $v$ from $V(G)$
            \State remove edges adjacent to $v$ from $E(G)$
            \State $count \leftarrow count+1$
        \EndIf
    \EndFor
    \If{$count = 0$}
        \State break.
    \EndIf
\EndWhile
\end{algorithmic}
\end{algorithm}

\subsection{Q3: Merger-Tree}

